\documentclass[10pt, twocolumn]{article}
\usepackage[margin=1in]{geometry}                % See geometry.pdf to learn the layout options. There are lots.
\geometry{letterpaper}                   % ... or a4paper or a5paper or ... 
%\geometry{landscape}                % Activate for for rotated page geometry
%\usepackage[parfill]{parskip}    % Activate to begin paragraphs with an empty line rather than an indent
\usepackage{graphicx}
\usepackage{amssymb}
\usepackage{epstopdf}
\DeclareGraphicsRule{.tif}{png}{.png}{`convert #1 `dirname #1`/`basename #1 .tif`.png}

\title{Analyzing Encrypted Web Traffic with Single- and Multi-Class SVMs}
\author{Emily Stark}
\date{\today}                                          % Activate to display a given date or no date

\begin{document}
\maketitle
%\section{}
%\subsection{}

\section{Introduction}
Web users, especially those in regions with censored 
Internet access, often use encryption in conjunction 
with anonymizing proxies to conceal which website 
they are visiting. For example, a user whose government 
restricts access to \texttt{www.facebook.com} can choose a 
proxy in an uncensored region, and use that proxy to 
forward encrypted HTTP requests and responses to and 
from Facebook. In theory, the censoring entity cannot determine from 
the user's encrypted traffic that the user is visiting 
Facebook. However, encryption does not conceal the direction, 
size, and timing of the packets that a user's web browsing 
generates. In practice, this small amount of information can 
be enough to determine the website that the user is visiting.


\end{document}  